%% Generated by Sphinx.
\def\sphinxdocclass{report}
\documentclass[letterpaper,10pt,english]{sphinxmanual}
\ifdefined\pdfpxdimen
   \let\sphinxpxdimen\pdfpxdimen\else\newdimen\sphinxpxdimen
\fi \sphinxpxdimen=.75bp\relax

\PassOptionsToPackage{warn}{textcomp}
\usepackage[utf8]{inputenc}
\ifdefined\DeclareUnicodeCharacter
% support both utf8 and utf8x syntaxes
\edef\sphinxdqmaybe{\ifdefined\DeclareUnicodeCharacterAsOptional\string"\fi}
  \DeclareUnicodeCharacter{\sphinxdqmaybe00A0}{\nobreakspace}
  \DeclareUnicodeCharacter{\sphinxdqmaybe2500}{\sphinxunichar{2500}}
  \DeclareUnicodeCharacter{\sphinxdqmaybe2502}{\sphinxunichar{2502}}
  \DeclareUnicodeCharacter{\sphinxdqmaybe2514}{\sphinxunichar{2514}}
  \DeclareUnicodeCharacter{\sphinxdqmaybe251C}{\sphinxunichar{251C}}
  \DeclareUnicodeCharacter{\sphinxdqmaybe2572}{\textbackslash}
\fi
\usepackage{cmap}
\usepackage[T1]{fontenc}
\usepackage{amsmath,amssymb,amstext}
\usepackage{babel}
\usepackage{times}
\usepackage[Bjarne]{fncychap}
\usepackage{sphinx}

\fvset{fontsize=\small}
\usepackage{geometry}

% Include hyperref last.
\usepackage{hyperref}
% Fix anchor placement for figures with captions.
\usepackage{hypcap}% it must be loaded after hyperref.
% Set up styles of URL: it should be placed after hyperref.
\urlstyle{same}
\addto\captionsenglish{\renewcommand{\contentsname}{Rechnungswesen (RW)}}

\addto\captionsenglish{\renewcommand{\figurename}{Fig.}}
\addto\captionsenglish{\renewcommand{\tablename}{Table}}
\addto\captionsenglish{\renewcommand{\literalblockname}{Listing}}

\addto\captionsenglish{\renewcommand{\literalblockcontinuedname}{continued from previous page}}
\addto\captionsenglish{\renewcommand{\literalblockcontinuesname}{continues on next page}}
\addto\captionsenglish{\renewcommand{\sphinxnonalphabeticalgroupname}{Non-alphabetical}}
\addto\captionsenglish{\renewcommand{\sphinxsymbolsname}{Symbols}}
\addto\captionsenglish{\renewcommand{\sphinxnumbersname}{Numbers}}

\addto\extrasenglish{\def\pageautorefname{page}}

\setcounter{tocdepth}{1}



\title{BWM Documentation Documentation}
\date{Oct 04, 2018}
\release{1.0}
\author{Fabian Rossmann}
\newcommand{\sphinxlogo}{\vbox{}}
\renewcommand{\releasename}{Release}
\makeindex
\begin{document}

\pagestyle{empty}
\maketitle
\pagestyle{plain}
\sphinxtableofcontents
\pagestyle{normal}
\phantomsection\label{\detokenize{index::doc}}


Das ist meine Zusammenfassung von BWM


\chapter{Externe Unternehmensrechnung I}
\label{\detokenize{pool1:externe-unternehmensrechnung-i}}\label{\detokenize{pool1::doc}}
Das ist meine Zusammenfassung von BWM


\section{System der doppelten Buchhaltung}
\label{\detokenize{pool1:system-der-doppelten-buchhaltung}}

\subsection{Begriff}
\label{\detokenize{pool1:begriff}}

\subsection{Funktionen und Teilbereiche des Rechnungswesens}
\label{\detokenize{pool1:funktionen-und-teilbereiche-des-rechnungswesens}}
fff


\subsection{Gesetzliche Bestimmungen hinsichtlich Buchführungspflicht und Formvorschriften}
\label{\detokenize{pool1:gesetzliche-bestimmungen-hinsichtlich-buchfuhrungspflicht-und-formvorschriften}}
fff


\subsection{System der doppelten Buchführung mit Bilanz als Ausgangspunkt}
\label{\detokenize{pool1:system-der-doppelten-buchfuhrung-mit-bilanz-als-ausgangspunkt}}
fff


\subsection{Kontenlehre und doppelte Erfolgsermittlung}
\label{\detokenize{pool1:kontenlehre-und-doppelte-erfolgsermittlung}}
fff


\subsection{Merkmale der doppelten Buchführung}
\label{\detokenize{pool1:merkmale-der-doppelten-buchfuhrung}}
fff


\bigskip\hrule\bigskip



\section{Verbuchnung laufender Geschäftsfälle 1}
\label{\detokenize{pool1:verbuchnung-laufender-geschaftsfalle-1}}

\subsection{Buchungen auf Lieferanten- und Kundenkonten}
\label{\detokenize{pool1:buchungen-auf-lieferanten-und-kundenkonten}}
\begin{DUlineblock}{0em}
\item[] \sphinxstylestrong{Kundenkonten}
\item[] Kundenkonten sind Forderungskonten
\item[] Schema: 20XXX
\end{DUlineblock}

\begin{DUlineblock}{0em}
\item[] \sphinxstylestrong{Lieferantenkonten}
\item[] Lieferantenkonten sind Verbindlichkeitskonten
\item[] Schema: 33XXX
\end{DUlineblock}


\subsection{Verbuchung von Wareneinkäufen, Warenverkäufen}
\label{\detokenize{pool1:verbuchung-von-wareneinkaufen-warenverkaufen}}
\sphinxstylestrong{Wareneinkauf}


\begin{savenotes}\sphinxattablestart
\centering
\begin{tabulary}{\linewidth}[t]{|T|T|T|}
\hline
&
5010
&
HW-Einsatz
\\
\hline&
2500
&
Vorsteuer
\\
\hline
an
&
33…
&
Lieferantenkonto
\\
\hline
\end{tabulary}
\par
\sphinxattableend\end{savenotes}

\sphinxstylestrong{Warenverkauf}


\begin{savenotes}\sphinxattablestart
\centering
\begin{tabulary}{\linewidth}[t]{|T|T|T|}
\hline
&
22…
&
Kundenkonto
\\
\hline
an
&
4000
&
HW-Erlöse
\\
\hline
an
&
3500
&
Umsatzsteuer
\\
\hline
\end{tabulary}
\par
\sphinxattableend\end{savenotes}


\subsection{Transportkosten}
\label{\detokenize{pool1:transportkosten}}
\begin{DUlineblock}{0em}
\item[] \sphinxstylestrong{Bezugskosten:}
\item[] Bezugskosten sind diejenigen Kosten, die bei der Beschaffung von Material oder Fertigerzeugnissen anfallen.
\end{DUlineblock}


\begin{savenotes}\sphinxattablestart
\centering
\begin{tabulary}{\linewidth}[t]{|T|T|}
\hline
&
Einkaufspreis
\\
\hline
+
&
Bezugskosten
\\
\hline
=
&
Einstandspreis
\\
\hline
\end{tabulary}
\par
\sphinxattableend\end{savenotes}

\begin{DUlineblock}{0em}
\item[] \sphinxstylestrong{Versandkosten:}
\item[] Unter Versandkosten versteht man jene Kosten, die durch die Versendung von Waren an den Kunden entstehen.
\item[] 
\item[] Verbuchung von Frachten:
\end{DUlineblock}


\begin{savenotes}\sphinxattablestart
\centering
\begin{tabulary}{\linewidth}[t]{|T|T|T|}
\hline
&
7300
&
Ausgangsfrachten
\\
\hline&
2500
&
Vorsteuer
\\
\hline
an
&
2800
&
Bank
\\
\hline
\end{tabulary}
\par
\sphinxattableend\end{savenotes}

\begin{DUlineblock}{0em}
\item[] Paketporto:
\item[] Über 10 KG und EMS-Sendungen
\end{DUlineblock}


\begin{savenotes}\sphinxattablestart
\centering
\begin{tabulary}{\linewidth}[t]{|T|T|T|}
\hline
&
7310
&
Paketgebühren 20\%
\\
\hline&
2500
&
Vorsteuer
\\
\hline
an
&
2700
&
Kassa
\\
\hline
\end{tabulary}
\par
\sphinxattableend\end{savenotes}

Briefe bis 2kg und Pakete bis 10kg


\begin{savenotes}\sphinxattablestart
\centering
\begin{tabulary}{\linewidth}[t]{|T|T|T|}
\hline
&
7311
&
Paketgebühren 0\%
\\
\hline
an
&
2700
&
Kassa
\\
\hline
\end{tabulary}
\par
\sphinxattableend\end{savenotes}


\subsection{Gutschriften aufgrund von Warenrücksendungen}
\label{\detokenize{pool1:gutschriften-aufgrund-von-warenrucksendungen}}
\sphinxstylestrong{Warenrücksendung an Lieferanten:}


\begin{savenotes}\sphinxattablestart
\centering
\begin{tabulary}{\linewidth}[t]{|T|T|T|}
\hline
&
33…
&
Liefernatenkonto
\\
\hline
an
&
5010
&
HW-Einsatz
\\
\hline
an
&
2500
&
Vorsteuer
\\
\hline
\end{tabulary}
\par
\sphinxattableend\end{savenotes}

\sphinxstylestrong{Warenrücksendung von Kunden:}


\begin{savenotes}\sphinxattablestart
\centering
\begin{tabulary}{\linewidth}[t]{|T|T|T|}
\hline
&
4000
&
HW-Erlöse
\\
\hline&
3500
&
Umsatzsteuer
\\
\hline
an
&
20…
&
Kundenkonto
\\
\hline
\end{tabulary}
\par
\sphinxattableend\end{savenotes}


\subsection{Nachträglich gewährter Rabatte}
\label{\detokenize{pool1:nachtraglich-gewahrter-rabatte}}
\begin{DUlineblock}{0em}
\item[] \sphinxstylestrong{Arten von Rabatten}
\end{DUlineblock}
\begin{itemize}
\item {} 
Skonto

\item {} 
Barzahlungsrabatt

\item {} 
Mengenrabatt

\item {} 
Sonderrabatt

\item {} 
Treuerabatt

\end{itemize}

\begin{DUlineblock}{0em}
\item[] \sphinxstylestrong{Nachträglicher Rabatt von Lieferanten}
\end{DUlineblock}


\begin{savenotes}\sphinxattablestart
\centering
\begin{tabulary}{\linewidth}[t]{|T|T|T|}
\hline
&
33…
&
Lieferantenkonto
\\
\hline
an
&
5010
&
HW-Einsatz
\\
\hline
an
&
2500
&
Vorsteuer
\\
\hline
\end{tabulary}
\par
\sphinxattableend\end{savenotes}

\begin{DUlineblock}{0em}
\item[] \sphinxstylestrong{Nachträglicher Rabatt von Lieferanten}
\end{DUlineblock}


\begin{savenotes}\sphinxattablestart
\centering
\begin{tabulary}{\linewidth}[t]{|T|T|T|}
\hline
&
4400
&
Erlösberichtigung
\\
\hline&
3500
&
Umsatzsteue
\\
\hline
an
&
20…
&
Kundenkonto
\\
\hline
\end{tabulary}
\par
\sphinxattableend\end{savenotes}


\subsection{Belegwesen}
\label{\detokenize{pool1:belegwesen}}
\begin{DUlineblock}{0em}
\item[] \sphinxstylestrong{Begriff:}
\item[] Der Beleg ist eine schriftliche Aufzeichnung (Dokument) über einen betrieblichen Vorgang, der alle wesentlichen Daten eines Geschäftsfalles enthält. Aufgrund seiner betrieblichen Relevanz muss er in der Buchhaltung ordnungsgemäß erfasst werden und dient als Grundlage für die Verbuchung des zugrunde liegenden Geschäftsfalles.
\item[] 
\item[] \sphinxstylestrong{Beleggrundsätze}
\end{DUlineblock}
\begin{itemize}
\item {} 
Keine Buchung ohne Beleg! Kein Beleg ohne Buchung!

\item {} 
Belege sind eindeutig mit Buchstaben der Beleggruppe zu kennzeichnen und hinsichtlich ihres chronologischen Verlaufs zu nummerieren.

\item {} 
Belege sind wie Urkunden zu behandeln (besondere Sorgfaltspflicht)

\item {} 
Auf den Belegen sind die Konten anzugeben, auf die gebucht werden soll (Vorkontierung)

\item {} 
Buchungsvermerk nach erfolgter Verbuchung (Abhaken oder Unterschrift)

\item {} 
Aufbewahrungspflicht von 7 Jahren in Österreich, 10 Jahre in Deutschland und der Schweiz

\end{itemize}

\begin{DUlineblock}{0em}
\item[] 
\item[] \sphinxstylestrong{Belegarten}
\end{DUlineblock}
\begin{itemize}
\item {} 
Eingangsrechnung - ER

\item {} 
Ausgangsrechnung - AR

\item {} 
Kassabelege - K

\item {} 
Bankbelege - B

\item {} 
PSK-Belege - PSK

\item {} 
Privat - P

\item {} 
Sonstige Belege - S

\end{itemize}

\begin{DUlineblock}{0em}
\item[] 
\end{DUlineblock}


\chapter{Pool1}
\label{\detokenize{pool2:pool1}}\label{\detokenize{pool2::doc}}
Das ist meine Zusammenfassung von BWM


\chapter{Pool1}
\label{\detokenize{pool3:pool1}}\label{\detokenize{pool3::doc}}
Das ist meine Zusammenfassung von BWM


\chapter{Pool1}
\label{\detokenize{pool4:pool1}}\label{\detokenize{pool4::doc}}
Das ist meine Zusammenfassung von BWM


\chapter{Pool1}
\label{\detokenize{pool5:pool1}}\label{\detokenize{pool5::doc}}
Das ist meine Zusammenfassung von BWM


\chapter{Pool1}
\label{\detokenize{pool6:pool1}}\label{\detokenize{pool6::doc}}
Das ist meine Zusammenfassung von BWM



\renewcommand{\indexname}{Index}
\printindex
\end{document}